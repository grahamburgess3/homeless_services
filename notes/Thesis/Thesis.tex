\documentclass[12pt,a4paper]{article}
\usepackage[left=20mm, right=20mm, top=20mm]{geometry} % to set up page formatting
\usepackage[skip=10pt]{parskip} % spacing in between paragraphs
\usepackage{graphicx} % Required for inserting images
\usepackage{amsmath} % Required for flexibility in mathematical equations
\usepackage{amssymb} % Required for certain math symbols e.g. E[.]
\usepackage{natbib} % Required for bibliography and citations
\usepackage{enumitem} % Required to remove gap between items in list
\usepackage{tikz} % Required to build tikz diagrams
\usepackage{xcolor} % to access colors in tikz diags
\usepackage{hyperref} % for web links
\usepackage{algorithm} % for algorithms
\usepackage{algpseudocode} % for algorithmics

\title{PhD Thesis}
\author{Graham Burgess}
\date{2027}

\begin{document}
%
\maketitle

\section{Literative Review}

If you include a section on `capacity planning in healthcare and homeless settings' as you did in Section 2.1 of 10 month review document, then take on board the following comments from Dave:

Section 2.1:
It is not entirely clear how you view the following ‘dimensions’ of capacity planning in the healthcare and homeless settings: 
a)	time horizon,
b)	discrete versus continuous time, 
c)	uncertainty in parameter values (e.g. mean service time, arrival rate) versus stochastic variability in service times and or arrival times of individuals,
d)	‘always busy’ versus ‘sometimes idle’ servers,
e)	‘average’ performance measures versus ‘stochastic’ performance measures, e.g. E(no people in housing) versus 90th percentile of no. in housing.  
The different references you refer to and the formulations you describe in greater detail are suitable for different combinations of the above.


\end{document}