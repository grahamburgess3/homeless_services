\documentclass[12pt,a4paper]{article}

\usepackage[left=20mm, right=20mm, top=20mm]{geometry} % to set up page formatting
\usepackage[skip=10pt]{parskip} % spacing in between paragraphs
\usepackage{amsmath} % Required for flexibility in mathematical equations
\usepackage{amssymb} % Required for certain math symbols e.g. E[.]
\usepackage{natbib} % Required for bibliography and citations
\usepackage{enumitem} % Required to remove gap between items in list
\usepackage{algorithm} % for algorithms
\usepackage{algpseudocode} % for algorithmics

\title{Suggestions for multi-fidelity R-SPLINE}
\author{Graham Burgess}
\date{December 2024}

\begin{document}
%
\maketitle

The retrospective search with piecewise-linear interpolation and neighborhood enumeration, R-SPLINE \citep{wang2013integer}, is a sample-average approximation procedure for discrete problems. Each iteration of R-SPLINE solves a sample-path problem retrospectively (R) using the SPLINE procedure. It is retrospective in the sense that each iteration of SPLINE uses the solution from the previous iteration as a `warm-start'. The SPLINE procedure repeatedly performs a search with piecewise linear interpolation (SPLI) and a neighbourhood enumeration (NE). The SPLI procedure performs a series of line searches. Each line search begins with gradient estimation at a point on the continuous domain, using piecewise linear interpolation (PLI). The SPLI procedure ends with a new point on the integer domain, along with a estimation of its objective value (estimated using multiple simulation replications). 

PLI estimates the gradient at a point in the continuous domain by estimating the objective value at all $d + 1$ integer points on the surrounding simplex. The gradient in a given co-ordinate direction is the difference between estimated objective values at opposite sides of the simplex, in that co-ordinate direction. Gradients are normalised when used in the line search. Objective value estimation at integer points, performed in line 8 of Algorithm 3 in R-SPLINE \citep{wang2013integer}, is done with multiple simulation replications, which could be computationally expensive. Our main suggestion is simple - in this part of the PLI procedure, we replace the multiple simulation replications with a single run of a low-fidelity model. In the homeless care system \citep{burgess2024time}, this low-fidelity model could be a fluid model.

We propose the following in case bias in the low-fidelity model leads to poor gradient estimates:
%
\begin{itemize}[noitemsep]
\item At the start of the R-SPLINE procedure, once an initial solution has been set, we could fit a meta model of the objective function response using some initial simulation effort. This meta model need only be a local model given that R-SPLINE finds locally optimal solutions. This meta model need only be defined on the integer-ordered domain, however a continuous approximation would still be useful if this was easier to fit. An appropriate meta model may consist of a `physical' term using a low-fidelity model and a corrective term (see for example \citet{osorio2015computationally})
  \item We could use this meta model in PLI for objective value estimation at integer points.
  \item At the end of each run of SPLI, simulation effort is used to estimate the objective value at the new integer point. We could use this information to update the meta model.
  \item For the first gradient estimation in each SPLI step, we could use both simulation and the meta model. We could evaluate the quality of the meta model-based gradient estimates and only proceed with it in the given SPLI step if the quality was sufficient. 
  \end{itemize}
  

\bibliographystyle{apalike}
\bibliography{bibliography.bib}

\end{document}