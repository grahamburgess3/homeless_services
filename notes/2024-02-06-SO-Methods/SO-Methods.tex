\documentclass{article}

\usepackage{graphicx} % Required for inserting images
\usepackage{amsmath} % Required for flexibility in mathematical equations
\usepackage{amssymb} % Required for certain math symbols e.g. E[.]

\title{SO methods}
\author{Graham Burgess}
\date{February 2024}

\begin{document}

\maketitle

\section{SAA}
``A Guide to Sample-Average Approximation'' - Kim, Ragu, Shane Henderson
Multi-dimensional newsvendor problem analysis shows that differentiability and expectation are interchangeable. The true function and the sample problem have the same nice properties of smoothness and concavity. So then - sufficiently large N -> effectively concave and deterministic opt can be used to find optimum.

We can say that SAA is appropriate for a problem if there is a structure to the sample average function which allows a nice deterministic optimisation solver to be used, and that the true function can be said to share that same structure.

It is also noted here

\section{Integer ordered decision vars}
``Simulation-based optimization over discrete setswith noisy constraints'' YAO LUO and EUNJI LIM

Ragu et al - RSpline

``CONSTRAINED OPTIMIZATION OVER DISCRETE SETS VIA SPSA WITH APPLICATION TO NON-SEPARABLE RESOURCE ALLOCATION'' Witney et al.

Other references from Ragu et al. - COMPASS and DSA (Discrete Stochastic Approx.)

\end{document}