\documentclass{article}

\usepackage{graphicx} % Required for inserting images
\usepackage{amsmath} % Required for flexibility in mathematical equations
\usepackage{amssymb} % Required for certain math symbols e.g. E[.]

\title{SO methods}
\author{Graham Burgess}
\date{February 2024}

\begin{document}

\maketitle

\section{Introduction}

How we categorise simulation optimisation (SO) methods:

\begin{itemize}
  \item decision variable discrete or continuous. In the discrete case: 
  \begin{itemize}
    \item Finite or infinite number of feasible solutions
    \item Ordered or unordered feasible solutions
    \item Random search used or not. For random search methods:
    \begin{itemize}
      \item neighbourhood structure (adaptive usually better than fixed)
    \end{itemize}
  \end{itemize}
  \item how the problem may be constrained:
  \begin{itemize}
    \item unconstrained, partially constrained, fully constrained
    \item stochastic constraints, deterministic constraints, box constraints
  \end{itemize}
  \item dimensionality of the problem
  \item local or global solutions found
  \item convergence guarentee

\end{itemize}

The problem we face in the housing waiting-list problem has the following corresponding characteristics:
\begin{itemize}
  \item In reality the decision variables (numbers of units of housing/shelter to build) are discrete, however we may work with a continuous approximation of the system. In the discrete case:
  \begin{itemize}
    \item We have a finite but large number of feasible solutions
    \item Within each dimension of the solution space, solutions are ordered, however ordering across dimensions is not obvious
  \end{itemize}
  \item The problem we currently have in mind is:
  \begin{itemize}
    \item fully constrained
    \item deterministic constraints  
  \end{itemize}
  \item There is potentially a medium number (approx. 10) of dimensions (how much of housing/shelter to build each year over some reasonable time horizon for planning
  \item We would like a global solution, however we may still be interested in locally convergent methods provide we can add an element of random searching to escape any local optimum if necessary. 
\end{itemize}

\newpage

\section{Integer ordered decision variables}

This occurs when the decision variables are numerical in nature and only take integer values (e.g. how many staff to hire, how many houses to build) - can be seen as the integer points on a lattice.

\subsection{Discrete Stochastic Approximation}

The idea is that the objective function on the discrete space (say $f$) is extended to the continuous space (say $\tilde{f}$) and SA is performed on this extension. Therefore at each iteration, the step to a new solution is not restricted to an integer point, which means it is 'free' to quickly move towards the optimal solution. Finally, one rounds to the nearest integer point. \newline


In the 1-D case, a simple linear interpolation between points in $f$ is performed to construct $\tilde{f}$. Then, following an initial guess $\theta_0$ of the minimiser of $\tilde{f}$, gradient descent steps are taken using a gradient estimator obtained via (multiple replications of) simulation of the integer points either side of $\theta_0$. The step sizes are a sequence of positive numbers. \newline

When $d>1$, the extension from $f$ to $\tilde{f}$ for $d>1$ is done by partitioning the continuous space (using a colletion of convex hulls built on points in the integer space) so that every point in the continuous space is contained within one of the convex hulls in the collection. Then $\tilde{f}$ is constructed as a linear combination of $f$ at all points on the surrounding convex hull. The weights in the linear combination depend on the proximity of the point in continuous space, to the corresponding points on the convex hull. Gradients are constructed using estimates (via multiple replications of simulation) of $f$ at the points on the convex hull. At each iteration, once a gradient is constructed, one moves to a new point by taking a step along the gradient, with the step sizes a sequence of positive numbers. Now that the dimensionality of the problem is greater than 1, multimodularity of the objective function (or $L^{\text{sharp}}$-convex) (as defined in the paper) is required to guarantee the convergence of this algorithm, sinc for multimodular (or $L^{\text{sharp}}$-convex) functions, the extenstion to the continuous space gives a convex function. \newline

Step-size at iteration $n$ is of the form $a/(b+n)$ where $a$ and $b$ are positive constants. $b$ is chosen as the total number of iterations and $a$ is selected by the user based on what is most advantagous for an initial step-size. 

\subsection{R-SPLINE}

R-SPLINE is ``retrospective'' in that it solves a series of sample-path problems, using the previous solution (i.e. retrospective) as a starting point for each new sample-path problem. Each sample-path problem is generated implicitly (i.e. sample-paths are collected as needed as the algorithm is progresses). Each sample-path problem is solved using a repeating combination of a continuous line search, and a neighbourhood evaluation. \newline

The line search moves along ``phantom'' gradients from integer point to integer point in search of a better solution. Each move starts with a random perturbation from the current best (integer) solution to a continuous point. Then the objective function value and gradient at this point is computed with a linear interpolation very similar to that in DSA. There is then ``continuous'' movement along this gradient and a jump to the nearest integer point, in search of an improved integer solution. The step-size at each continious movement is calculated to be as small as possible so that the jump to the nearest integer point will land on a new integer point. The line search stops when a move within the search is small (below some threshold). Once a line search is complete, a neighbourhood evaluation is performed to check that the current best is indeed a local solution (within all of its nearest neighbours, in every dimension). The combination of the line search and the neighbourhood evaluation is repeated until no improvement is found, or until the total simulation budget is used. \newline

There is similarity between R-SPLINE and DSA in the way the objective function value and gradient in the continous space are calculated. However, there are differences: DSA, once transformed into the continous space, only moves in the continuous space, whereas R-SPLINE regularly jumps to the nearest integer point, before randomly perturbing away from that integer point. R-SPLINE involves repetitions of the line search and a neighbourhood evaluation, however DSA performs no neighbourhood evaluation as it moves towards the optimal solution. There is also a difference in how the step size is calculated as the search moves along the gradients. R-SPLINE has has an adaptive step-size (explained above), whereas DSA gets smaller and smaller as the algorithm progresses. Both algorithms increase the number of simulation replications from iteration to iteration. \newline

More recently, a new algorithm called ``ADALINE'' incorporates many features of R-SPLINE but invovles an adaptive sample size at each iteration so as not to waste computation budget. 

\subsection{COMPASS}


Convergent optimisation via most-promising-area stochastic search (COMPASS). Based on random search (like many other DOvS algs in literature) but offers a unique adaptive neighbourhood structure known as the ``Most promising area'' (MPA). For terminating and steady state simulation. For fully/partially/un- constrained problems. The fully-constrained setting is described as follows: \newline

The algorithm starts by simulating an initial solution and the entire feasible region is in the MPA. Then, new solutions are sampled from the MPA and simulated. Once simulation has taken place, the ``promising''-index of each feasible solution is the sample mean performance of the visited solution which is closest to it. Therefore, the MPA is the set of feasible sols which have the current sample best solution as their closest visited solution. By design, the only previously-visted solution in the MPA is the current best. At each iteration, COMPASS updates the current best solution and the MPA and assigns simulation experiments (using a Simulation Allocation Rule) to candidate solutions sampled uniformly from the the MPA. It keeps information from previously-visited solutions over the course of the algorithm.   \newline

When the problem is partially- or un-constrained, a boundary around the current best solution is maintained and is widened as candidate solutions get close to the boundary. Within the boundary, the algorithm works as in the above fully-constrained description. \newline

If the algorithm gets stuck in the wrong area (due to simulation noise) because all visited solutions are allocated further simulation budget, the algorithm will eventually move away from the sub-optimal area. The user must think carefully about how many solutions to sample from the MPA at each iteration. Fewer will reduce the time taken to find a local optimum, but the quality of this local optimum may not be good. 

\subsection{GMRF}

Model execution is too slow to model even a fraction of the optimal solutions. Ranking and selection typically simulate all and then reach some guarantee of PCS (frequentist). Adaptive Random Search if its impossible to simulate everything. this paper looks for R+S style guarantees with an efficient search, learning the spatial correlations along the way. Multi-resolution framework: region-level learning and solution-level learning. \newline

The precision matrix is the inverse of the covariance matrix for a random vector. The diagonals are 'conditional precisions' - i.e. the inverse of the conditional variances (the variances given knowledge of all other elements of the random vector). The non-diagonal elements are 'conditional covariance' - i.e. the covariance given knowledge of all other elements of random vector and can be normalised to 'conditional correlation'. \textbf{Unfinished: } Can accept these meanings of the precision matrix intuitively, but have not studied a proof. \newline

The neighbourhood structure of a graph can define the non-zero entries of a precision matrix. The authors introduce parameters $\boldsymbol{\theta}$ where $\theta_0$ is the conditional precision of each solution and $\theta_j$ is the conditional correlation if two solutions differ by one unit in the $j$th dimension. This means response surfaces can change more rapidly in one direction compared to another. $\boldsymbol{\theta}$ entries are all between zero and one, we want positive correlations but should be below one. These parameters are estimated with MLE. \newline

So $\boldsymbol{y}$ is true objective function (objective function is the expectation random $Y$). $\boldsymbol{y}$ is unknown but we model it as $\mathbb{Y}$, a realisation of a GMRF as defined. What we observe is realisations of $Y$ which is modelled as the GMRF + noise. We obtain sample means which we also model as GMRF + noise. This latter noise is modelled as Gaussian with diagonal precision matrix if CRN not used. Interested in the conditional distribution of $\mathbb{Y}$ given the simulation output at the design points (feasible solutions which have been simulated). Given a vector of sample means at the design points, the authors prove a conditional distribution of $\mathbb{Y}$. \newline

Optimisation performed using Complete Expected Improvement which averages over the both full conditional joint distribution of $\mathbb{Y}$ and the simulation noise. EI estimates the optimality gap given the current data (Knowledge gradient looks at the predictive distribution from simulating other solutions more and picks that which predicts the best improvement - useful when more simulation is expensive and you want to see if its worth it. \newline

\textbf{Question: } Claim is that the average is over the full conditional joint dist of $\mathbb{Y}$ and the simulation noise, but isn't the latter incorporated into the former in the conditional distribution of the former? Is the simulation noise maybe also explicitly acknowledged because the CEI takes a difference between two random variables (obj val at current best and add proposed)? \newline

The solution-level algorithm is what I'd expect, except interesting to note that after the candidate solution is found which maximises the CEI, extra simulation budget is expended on \emph{both} that solution \emph{and} the current best, before proceeding. \newline

At each computation of the CEI (note: this computation is needed for n-1 solutions at every iteration of the algorithm) - the conditional distribution $\mathbb{Y}$ given the data is estimated, using estimates of parameters within the matrix. The computationally expensive bit is inverting the conditional precision matrix. But because it is sparse (due to neighbourhood structre and Markov property and due to the diagonal nature of the intrinsic noise matrix) and because only a small number of matrix elements are actually needed for the computation, matrix techniques which exploit these characteristics can be used. \newline

The convergence of this algorithm to the optimal solution as the number of sim reps goes to inf is based on the proof that with probability one, every solution will be simulated infinitely often given this algorithm.  \newline

A regional level approachis also given where there are GMRFs where each node corresponds to a region of the solution space, and there are also within-region solution-level GMRFs. This type of approaches is considered helpful for a problem with low dimensionality but a very large number of solutions. \newline


\subsection{RSM in a discrete setting}

Response Surface Optimization with Discrete Variables (2004)

\section{Continuous decision variables}

\subsection{Sample Average Approximation}

``A Guide to Sample-Average Approximation'' - Kim, Ragu, Shane Henderson
Multi-dimensional newsvendor problem analysis shows that differentiability and expectation are interchangeable. The true function and the sample problem have the same nice properties of smoothness and concavity. So then - sufficiently large N -> effectively concave and deterministic opt can be used to find optimum.

We can say that SAA is appropriate for a problem if there is a structure to the sample average function which allows a nice deterministic optimisation solver to be used, and that the true function can be said to share that same structure.

\subsubsection{Infinite dimension SO problems - (function decisions) - approx with SAA}

Singham et al. 

\subsubsection{Retrospective framework}

How this differs from SAA

\subsection{Stochastic Approximation}

\subsection{Meta-modelling}

\subsubsection{Trust-region methods}

\subsubsection{Response Surface Methodology}

STRONG

\subsubsection{Global meta models}

BO

\subsubsection{Multi-fidelity models}

A simulation-based optimization algorithm for dynamic large-scale urban transportation problems - Osorio and Chong 2018 \newline

Zirui Cao et al. CLUSTER-BASED SAMPLING ALLOCATION FOR MULTI-FIDELITY SIMULATION OPTIMIZATION (WSC 2023)

\end{document}