\documentclass{article}

\usepackage{graphicx} % Required for inserting images
\usepackage{amsmath} % Required for flexibility in mathematical equations
\usepackage{amssymb} % Required for certain math symbols e.g. E[.]

\title{SO methods}
\author{Graham Burgess}
\date{February 2024}

\begin{document}

\maketitle

\section{Introduction}

How we categorise simulation optimisation (SO) methods:

\begin{itemize}
  \item decision variable discrete or continuous. In the discrete case: 
  \begin{itemize}
    \item Finite or infinite number of feasible solutions
    \item Ordered or unordered feasible solutions
    \item Random search used or not
  \end{itemize}
  \item how the problem may be constrained:
  \begin{itemize}
    \item unconstrained, partially constrained, fully constrained
    \item stochastic constraints, deterministic constraints, box constraints
  \end{itemize}
  \item dimensionality of the problem
  \item local or global solutions found
  \item convergence guarentee

\end{itemize}

The problem we face in the housing waiting-list problem has the following corresponding characteristics:
\begin{itemize}
  \item In reality the decision variables (numbers of units of housing/shelter to build) are discrete, however we may work with a continuous approximation of the system. In the discrete case:
  \begin{itemize}
    \item We have a finite but large number of feasible solutions
    \item Within each dimension of the solution space, solutions are ordered, however ordering across dimensions is not obvious
  \end{itemize}
  \item The problem we currently have in mind is:
  \begin{itemize}
    \item fully constrained
    \item deterministic constraints  
  \end{itemize}
  \item There is potentially a medium number (approx. 10) of dimensions (how much of housing/shelter to build each year over some reasonable time horizon for planning
  \item We would like a global solution, however we may still be interested in locally convergent methods provide we can add an element of random searching to escape any local optimum if necessary. 
\end{itemize}

\section{Discrete decision variables}
``A Guide to Sample-Average Approximation'' - Kim, Ragu, Shane Henderson
Multi-dimensional newsvendor problem analysis shows that differentiability and expectation are interchangeable. The true function and the sample problem have the same nice properties of smoothness and concavity. So then - sufficiently large N -> effectively concave and deterministic opt can be used to find optimum.

We can say that SAA is appropriate for a problem if there is a structure to the sample average function which allows a nice deterministic optimisation solver to be used, and that the true function can be said to share that same structure.

It is also noted here

\section{Integer ordered decision vars}
``Simulation-based optimization over discrete setswith noisy constraints'' YAO LUO and EUNJI LIM

Ragu et al - RSpline

``CONSTRAINED OPTIMIZATION OVER DISCRETE SETS VIA SPSA WITH APPLICATION TO NON-SEPARABLE RESOURCE ALLOCATION'' Witney et al.

Other references from Ragu et al. - COMPASS and DSA (Discrete Stochastic Approx.)

\section{Continuous decision varialbes}

\end{document}