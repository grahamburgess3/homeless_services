\documentclass[12pt,a4paper]{article}

\usepackage[left=20mm, right=20mm, top=20mm]{geometry} % to set up page formatting
\usepackage[skip=10pt]{parskip} % spacing in between paragraphs
\usepackage{graphicx} % Required for inserting images
\usepackage{amsmath} % Required for flexibility in mathematical equations
\usepackage{amssymb} % Required for certain math symbols e.g. E[.]
\usepackage{natbib} % Required for bibliography and citations
\usepackage{enumitem} % Required to remove gap between items in list
\usepackage{tikz} % Required to build tikz diagrams
\usepackage{xcolor} % to access colors in tikz diags
\usepackage{hyperref} % for web links

\title{Multi-fidelity modelling and optimisation for long-term capacity planning: ten month PhD review (ammendments)}
\author{Graham Burgess}
\date{November 2024}

\begin{document}
%
\maketitle
%

In this document I outline proposed ammendments to the initial draft of the 10-month review document. These suggestions are based on our discussion on 22nd October. They do not include all the ammendments which follow on from feedback received by email. I also list some potential topics of conversation for Friday 8th November.

\section{Introduction}
%
\begin{itemize}[noitemsep]
\item Introduce the more general context of queueing models in long-term capacity planning, where the homeless care setting is one example. 
\item Introduce the distinction between long and short service times when discussing examples of long-term capacity planning
\end{itemize}
%
\section{Literature Review} \label{lit-rev}
%
\begin{itemize}[noitemsep]
\item include discussion of recent JOS paper on long-term bed modelling for critical care hosptial units
\item include more references in Section 2.2.1 (Overview of SO methods)
\item include GMIA extensions rapid GMIA and multi-fidelity GMIA (previously disccussed in Section \ref{mfso}). 
\item include discussion of current literature on multi-fidelity Bayesian optimisation.
\end{itemize}
%
\section{Models of multi fidelity} \label{models}

- 

\section{Deterministic optimisation with low-fidelity model} \label{do}

- 

\section{Discussion of uncertainty} \label{uncert}
%
\begin{itemize}[noitemsep]
\item discuss differences between problems with long and short service times: with long service times we are more interested in input uncertainty, and with short service times we are more interested in stochastic uncertainty.
\item expand upon ideas for how input uncertainty could be incorporated into a future MFSO algorithm.
\item discuss problem of SO in queueing settings: bad solutions have high stochastic uncertainty and therefore take a lot of simulation effort to eliminate. 
\end{itemize}
%
\section{Potential research contributions} \label{mfso}
\begin{itemize}
\item Expand upon and emphasize the suggested focus: to develop efficient MFSO methods for long-term capacity planning problems. To do this by building upon current MFSO methods and integer-ordered SO methods by incorporating information about the structure of these problems which is available from low-fidelity queueing models. 
\end{itemize}
%
\underline{To discuss on 8th November, 2024:}
\begin{itemize}[noitemsep]
\item The proposed PhD problem is `artificial' in the sense that in reality public-sector decision makers usually have a small number of potential plans for serious consideration. Discuss the implications of this on the proposed PhD direction.
\item The proposed next steps address rather technical challenges. Discuss whether this is sufficient for a PhD.
  \item A meeting plan for my time in Lancaster: who to meet when and at what frequency. 
\end{itemize}
%
\end{document}