\documentclass{article}

\usepackage{graphicx} % Required for inserting images
\usepackage{amsmath} % Required for flexibility in mathematical equations
\usepackage{amssymb} % Required for certain math symbols e.g. E[.]
\usepackage{natbib} % Required for bibliography and citations
\usepackage{enumitem} % Required to remove gap between items in list

\title{Modelling and optimising the housing of homeless populations: ten month PhD review}
\author{Graham Burgess}
\date{October 2024}

\begin{document}

\maketitle

\begin{abstract}
Modelling and optimisation are popular tools for supporting resourcing and capacity decisions in healthcare and homeless settings. We show how deterministic optimisation with a fluid flow model can support long-term capacity planning for a homeless care setting in the San Francisco Bay Area, California. Models of multi fidelity, including stochastic simulation, are available in this setting, and the solution space is integer-ordered. We therefore explore both multi fidelity and integer-ordered simulation optimisation methods and discuss potential research contributions at the intersection of these active fields of research.
\end{abstract}

\section{Introduction}

Homelessness is a growing problem faced by communities worldwide. An example is Alameda County in the San Francisco Bay area, California, where approximately $8000$ people experienced homelessness in $2021$ alone. Decision makers within these communities typically have some leverage over how relevant resources are allocated to help alleviate homelessness. In Alameda County, the local government must split their resources between emergency shelter and permanent social housing. Shelter is relatively cheap and quick to set up. It provides a safer alternative to street homelessness but does not provide a stable long-term living situation for its residents. Permanent social housing is more expensive than shelter and can take longer to set up, but it does offer the stable long-term living situationd that homelessness people need to improve their quality of life. Building capacity to alleviate homelessness takes time, especially since funds are typically available in varying amounts from year to year. Decision makers in communities like Alameda County must therefore make good time-varying capacity planning decisions to reduce homelessness now and in the future. \newline

Operational research (OR) methods offer helpful tools to support such public sector decision-making. Optimisation helps decision-making by looking for a feasible solution which performs best across a (potentially infinite) set of alternative feasible solutions. To do this, a model of the performance of a solution is needed, and the quality of the model affects the quality of the subsequent optimisation results. We can model the homelessness care system as a queue. Homeless people must wait for permanent social housing to become available. Some of those in the queue for social housing can be accommodated in shelter, while the rest will be street homeless. Once a housing `server' becomes availabe, a resident may stay their for some time (in some cases, the rest of their lives) before they ultimately leave the system. \newline

The most accurate model of this queueing system is a high-fidelity stochastic simulation model. In this case, one can only estimate the performance of a solution and the subsequent optimisation falls in the realm of simulation optimisation (SO). There are different SO methods for different types of problem (which we later discuss) but the issue of limited computational resources pervades all SO methods. This issue stems from from the fact that a stochastic simulation is typically computationally expensive to run, and many simulation replications are required to be confident of a solution's performance, given the associated uncertainty. \newline

As is common in queueing systems, lower fidelity models such as analytical queueing models offer a computationally cheaper alternative to stochastic simulation. They also offer helpful alternative perspectives on the dynamics of a queueing system. The drawback is that these models are typically less accurate, given the necessary assumptions which must be made. If one only uses a low-fidelity deterministic model to evaluate the performance of a solution, optimisation falls in the realm of deterministic optimisation. Performing this deterministic optimisation can be a helpful first step in the decision-making process. However, there is more we can do with our low fidelity models. Multi-fidelity simulation optimisation (MFSO) enables low-fidelity models to be used alongside high fideltiy stocahstic simulation in a SO algorithm which reduces the computational burden and therefore enables an optimal solution to be found more efficiently. Novel MFSO methods will be the main topic of this PhD research. \newline

This document is organised as follows: in Section \ref{lit-rev} we briefly review relevant literature on modelling and optimisation in homeless care settings and healthcare settings. There are many similarities between these two and the latter is more widely studied in the literature. We also review relevant SO methods including MFSO. In Sections \ref{models} and \ref{do} we discuss the main content of the PhD research to date. Section \ref{models} introduces three models of multi-fidelity for homeless care systems. Section \ref{do} introduces an optimisation formualation which addresses the time-varying capacity planning problem and we solve this problem in a determinsitc setting. In Section \ref{uncert} we discuss how different types of uncertainty affect our decision-making process, and this discussion motivates the need for simulation optimisation. In Section \ref{mfso} we discuss relevant interesting gaps in the current MFSO literature which the remainder of this PhD research seeks to address.

\section{Literature Review} \label{lit-rev}

\subsection{Modelling and optimisation in healthcare settings}

Modelling hospital waiting lists using stochastic simulation e.g. \cite{wood2022supporting} and using stocks and flows e.g. \cite{worthington1991hospital}. Optimisation such as \cite{argyris2022fair} who balance efficiency and fairness in healthcare provision. 

\subsection{Modelling and optimisation in homeless care settings}

Simulation modelling of homeless care system in Alameda County \citep{singham2023discrete} and of shelters for runaway homeless youths (RHYs) \citep{kaya2022discrete}. Optimisation such as \cite{kaya2022improving} who minimise the cost of matching demand with supply for RHYs who require beds and support services.

\subsection{Simulation optimisation (SO)}

\subsubsection{Overview of SO methods}

\begin{itemize}[noitemsep]
\item Discrete SO: ranking \& selection, adaptive random search, integer-ordered.
\item Continuous SO: sample average approx, stochastic approx, meta models.
\end{itemize}

\subsubsection{Integer-ordered SO methods}
\begin{itemize} [noitemsep]
    \item Retrospective search with piecewise-linear interpolation and neighborhood enumeration (R-SPLINE) \citep{wang2013integer}.
    \item Discrete Stochastic Approximation \citep{lim2012stochastic}
    \item Gaussian Markov Random Fields \citep{l2019gaussian}
\end{itemize}

\subsubsection{Multi fidelity SO methods}

\begin{itemize}[noitemsep]
\item Using deterministic optimisation results to begin simulation optimisation e.g. \cite{jian2015introduction}.
\item Ordinal transformation with optimal sampling \citep{xu2016mo2tos}.
\item Modelling the error of a low-fidelity model
\begin{itemize}[noitemsep]
\item Polynomial error terms e.g. \cite{chong2018simulation}.
\item Gaussian Process error terms e.g. \cite{huang2006sequential}.
\end{itemize}
\item Multi-fidelity Expensive Black Box (Mf-EBB) Optimisation
\end{itemize}

\section{Models of multi-fidelity} \label{models}

\section{Deterministic optimisation with low-fidelity model} \label{do}

\begin{itemize}[noitemsep]
\item Optimisation formulations
\item Numerical results
\end{itemize}

\section{Discussion of uncertainty} \label{uncert}

\begin{itemize}[noitemsep]
\item Stochastic uncertainty in homeless care problem (arrival/service processes).
\item Input model uncertainty: good input models for today cannot reliably predict future events.
\end{itemize}

\section{Potential contributions at intersection of integer-ordered and multi fidelity SO} \label{mfso}

\begin{itemize}[noitemsep]
\item Using low-fidelity models to quickly compute gradients in RSPLINE/DSA.
\item Adding prior information to GMRF using low-fidelity model.
\item Modelling errors of low fidelity models using GMRF.
\end{itemize}

\section{Conclusion}\label{conc}

\newpage

\bibliographystyle{apalike}
\bibliography{bibliography.bib}

\end{document}